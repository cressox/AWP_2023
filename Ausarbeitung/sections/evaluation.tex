\subsection{Selbsttest zu Metriken und Framworks}
Zur Evaluation verschiedener Schwellwerte zur Blinzeldetektion und Frameworks wurde einen Selbsttest im Wachzustand durchgeführt, wobei zu zweit in einer alltäglichen Arbeitsumgebung mit natürlicher Beleuchtung über die interne Webcam des eigenen Laptops das Programm circa eine Minute lang ausgeführt wurde. Von diesen Versuchen entnahmen wir die selbstgezählte reale und die detektierte Anzahl der Blinzelschläge in diesem Zeitraum für verschiedene Parametervoreinstellungen. Zusätzlich führten wir dies für zwei verschiedene Frameworks aus, um durch Selbsttests herauszufinden, welches für unseren Anwendungsbereich geeigneter ist. Für Proband 1 ist zu erkennen, dass MediaPipe im Durchschnitt eine signifikant bessere Detektionsrate aufweisen kann als Dlib. Hierbei ist es wichtig, dass weder zu viele noch zu wenig Blinzelschläge detektiert werden. Da keine Toleranz für Fehldetektionen geplant war, ist dieser Wert zu optimieren. Für Probanden 2 sind ähnliche Ergebnisse zu verzeichnen, eine fehlerfreie Erkennung kann auch hier nur MediaPipe liefern. Jedoch ist zu bemerken, dass für diese Versuchsperson die optimale Detektion nur bei einem kleineren Schwellwert des EAR-Wertes gegeben ist. Zu begründen ist dies durch die generellen Unterschiede in Individuen sowie die persönliche Form und Anatomie des Auges [Verweis/Quelle]. Für jede Person ist der EAR-Wert im geöffneten Zustand verschieden, ein kleinerer Wert tendiert dazu, dass bei einem vergleichsweise hohen Schwellwert einige Blinzelschläge fälschlicherweise erkannt werden, da der Schwellwert bereits ohne ein Schließen des Auges unterschritten wird. Dies ist auch in dem Test aufgefallen, so wurden für einige Kombinationen zu viele Blinzelschläge detektiert, weshalb der Detektor nicht vertrauenswürdig wäre. Ein zu kleiner Schwellwert resultiert jedoch darin, dass aufgrund der möglicherweise geringen Framerate der Videos ein Blinzeln nicht erkannt wird, wenn nicht zufälligerweise ein Bild bei der vollständigen Schließung des Auges aufgenommen wird. Nach Beachtung dieser Zusammenhänge und Auswertung der Tabelle des Selbsttest ist der Entschluss zu ziehen, dass die Benutzung von MediaPipe als Framework das geeignetere Mittel ist. Weiterhin stellte sich der Schwellwert für die Blinzeldetektion von 0,16 als passend heraus, da dieser verschiedene Probleme minimiert. Für diesen Wert wurde die höchste Genauigkeit erreicht, da sowohl die Anzahl der falschpositiven als auch falschnegativen Werte für das ausgewählte Framework gleich Null ist. Somit ist zu sagen, dass die implementierte Detektion verlässlich und genau arbeitet. 

Tabelle einfügen

%Please add the following packages if necessary:
%\usepackage{booktabs, multirow} % for borders and merged ranges
%\usepackage{soul}% for underlines
%\usepackage[table]{xcolor} % for cell colors
%\usepackage{changepage,threeparttable} % for wide tables
%If the table is too wide, replace \begin{table}[!htp]...\end{table} with
%\begin{adjustwidth}{-2.5 cm}{-2.5 cm}\centering\begin{threeparttable}[!htb]...\end{threeparttable}\end{adjustwidth}
\begin{table}[!htp]\centering
	\caption{Selbsttest zu verschiedenen Schwellwerten und Frameworks}\label{tab: }
	\scriptsize
	\begin{tabular}{lrrrrr}%\toprule
		&\multicolumn{4}{c}{\textbf{Detektierte/reale Blinzelschläge}} \\\cmidrule{2-5}
		&\multicolumn{2}{c}{\textbf{Proband 1}} &\multicolumn{2}{c}{\textbf{Proband 2}} \\\cmidrule{2-5}
		\textbf{Blink Threshold} &\textbf{Dlib} &\textbf{MediaPipe} &\textbf{Dlib} &\textbf{MediaPipe} \\\midrule
		\multirow{2}{*}{0.16} &33/43 &45/45 &54/53 &54/54 \\
		&39/53 &58/58 &49/47 &61/61 \\
		\multirow{2}{*}{0.18} &38/48 &60/60 &49/47 &61/50 \\
		&33/52 &60/53 &57/52 &42/41 \\
		\multirow{2}{*}{0.20} &50/52 &64/64 &65/44 &60/37 \\
		&47/53 &57/57 &57/49 &70/45 \\
		& & & & \\
		\textbf{$\varnothing$ EAR} &0.32 &0.295 &0.259 &0.252 \\
		& & & & \\
		\textbf{$\varnothing$ PERCLOS} &0.08 &0.063 &0.109 &0.107 \\
		%\bottomrule
	\end{tabular}
\end{table}

Neben der Anzahl der Blinzelschläge wurden zusätzlich die durchschnittlichen EAR- und PERCLOS-Werte betrachtet, um eventuelle Zusammenhänge nachvollziehen zu können. Hierbei ist deutlich zu entnehmen, dass der EAR-Wert für Proband 2 deutlich kleiner und der PERCLOS-Wert deutlich größer als die des Probanden 1 sind. Dies spiegelt genau die Zusammenhänge der Metriken und der verschiedenen geeigneten Schwellwerts wider, da für Proband 2 ein kleinerer Schwellwert besser geeignet ist, was auf ein generell schwächer geöffnetes Auge hinweist. Dies würde in einem geringeren EAR-Wert resultieren, was die Daten verifizieren. Weiterhin sollte der PERCLOS-Wert höher sein, je größer die Anzahl an Blinzelschlägen oder je länger die Blinzeldauer pro Zeitspanne ist, da der Anteil, in dem die Augen geschlossen sind, größer ist. Unsere Beobachtungen stimmen mit diesem Verhalten überein, für Proband 1 wurde ein kleinerer Wert als für Proband 2 ermittelt.

\subsection{Vergleich verschiedener Klassifikatoren}
evtl Gesamtevaluation