Während des gesamten Erstellungsprozesses sind wir als Gruppe vor eine Herausforderungen geraten, an welchen wir stets wachsen und lernen konnten. Dieses Kapitel dient als eine Reflektion unserer Arbeitsweise, in dem wir aufzeigen, was für zukünftige Projekte erlernt wurde, um das eigene Vorgehen stets verbessern zu können. Weiterhin soll eine Darstellung möglicher Erweiterungen die Vielfältigkeit des gewählten Themengebietes aufzeigen und Inspiration für weitere Arbeiten geben.

\subsection{Lessons Learned}


Aufwand und Nutzen nie außer Acht lassen

Von zu optimistischer Zeitplanung absehen


\subsection{Future Work}
Um die Vielfältigkeit dieses Projektes aufzuzeigen und für weitere Arbeiten zu inspirieren, ist es sinnvoll, noch offene Themen und Erweiterungen zu beleuchten. Der erste nennenswerte Punkte ist ein gezieltes Erhöhen der Robustheit des Programms. Der für uns optimale Schwellwert zur Blinzeldetektion wurde empirisch ermittelt und war für unsere Anwendungsgebiet passend. Sollten man diese Voreinstellungen auf andere Ethnien testen, könnte es jedoch gegebenenfalls zu Komplikationen kommen. Um dem entgegenzuwirken, sollte der Schwellwert variabel gesetzt werden, indem ein optimaler relativer Threshold zum durchschnittlichen EAR-Wert der Person ermittelt wird. Um diesen zu ermitteln sind jedoch weitere Test und Forschung notwendig, jedoch ist dies eine Verbesserung der Anwendbarkeit des Programms.

Weiterhin kann die Wahl anderer Klassifikationsverfahren eine weitere Genauigkeitsverbesserung mit sich bringen. Hierfür sind verschiedenste Versuche notwendig, um aus vielen Klassifikatoren den bestgeeigneten ermitteln zu können, wir haben uns dabei aus aufwandsmindernden Gründen auf drei Möglichkeiten beschränkt. Die Wahl einer anderen Methode könnte weiterhin den Vorteil mit sich bringen, dass man unter zusätzlicher Vergrößerung des Datensatzes einen Score zur Müdigkeit entwickeln kann. Somit verfällt eine strikte Einteilung in vordefinierte Klassen und Veränderungen der Werte über bestimmte Zeitreihen sind besser nachzuvollziehen und zu analysieren.

Um unser Anwendungsgebiet des Autofahrens weiter zu vertiefen und die Sicherheit zusätzlich zu erhöhen, ist es möglich, nicht nur die Müdigkeit zu erkennen, sondern auch Unaufmerksamkeiten zu detektieren. Von hoher Relevanz hierbei kann ein erkennen der Zeit sein, in welcher der Fokus nicht auf den Verkehr gelegt wird. Somit erscheint eine Warnung, wenn sich der Blick zu lange dem Handy, Infotainmentsystem des Fahrzeugs oder in der weniger relevanten Fahrumgebung richtet.

Sollte eine solche Software professionelle Anwendung in einem Auto finden, so ist ein Einsatz auf sich im Auto befindender Hardware unausweichlich. Somit kann beispielsweise im Armaturenbrett eine kleine Webcam installiert sein, welche über das Boardsystem angeschlossen ist und die Warnungen können direkt über ein Cockpit, Head-Up Display und die interne Soundanlage abgespielt werden. Sollten diese Schritte zukünftig genauer bedacht werden, so würde dem Weg dieser Technolgie in Serienfahrzeuge nichts im Wege stehen und die Sicherheit aller Verkehrsteilnehmenden kann somit maßgeblich verbessert werden.

%geeignetere Evaluationsmehtodiken entwickeln für die Müdigkeit?





\subsection{Zusammenfassung}