Während des gesamten Erstellungsprozesses sind wir als Gruppe vor einige Herausforderungen geraten, an welchen wir stets wachsen und lernen konnten. Dieses Kapitel dient als eine Reflektion unserer Arbeitsweise, in dem wir aufzeigen, was für zukünftige Projekte erlernt wurde, um das eigene Vorgehen stets verbessern zu können. Weiterhin soll eine Darstellung möglicher Erweiterungen die Vielfältigkeit des gewählten Themengebietes aufzeigen und Inspiration für weitere Arbeiten geben.

\subsection{Lessons Learned}
Während des gesamten Erstellungsprozesses sind uns als Gruppe einige Punkte bewusst geworden, welche wir bei einer nächsten Zusammenarbeit gesondert beachten würden. Der erste Punkt ist in der Planungsphase anzusiedeln, genauer in der zeitlichen Einteilung. Uns wurde im Laufe der Entwicklungen bewusst, dass mehr Zeit für Fehler und zu behebende Probleme eingeplant werden muss. Wir planten zu optimistisch und rechneten mit deutlichen weniger Schwierigkeiten, was uns bedingt in Stress verfallen lassen hat. Bei der nächsten Planung ist demzufolge nicht zu optimistisch heranzugehen, es muss mehr Kapazität für unerwartete Fehler eingeplant werden, welche nicht in kürzester Zeit behoben werden können, um trotzdem noch den zeitlichen Rahmen einhalten zu können.

Wird sich tiefer in die Müdigkeitserkennung eingearbeitet, so fällt schnell die Vielfältigkeit des Themas auf. Ideen, die anfangs gesammelt wurden und umgesetzt werden sollten, sind nach ausführlicher Recherche nur noch ein kleiner Teil eines großen Kontingents an Möglichkeiten. Davon haben wir uns anfangs stark beeinflussen lassen und wir wollten jede Möglichkeit austesten, um teilweise auch nur marginale Verbesserungen zu erreichen. Später ist uns die Relevanz eines klar definierten Fokus bewusst geworden. Hierbei sollten bei der Recherche und dem Prüfen verschiedener Erweiterungen stets Aufwand und Nutzen gegenübergestellt werden. Da wir dies zuerst missachteten, verbrachten wir viel Zeit damit, verschiedene Methoden zu prüfen anstatt sich gemeinsam festzulegen und dies weiter zu vertiefen. Um den Zeitplan und Umfang des Projektes einzuhalten, ist demzufolge stets ein klarer Fokus und eine Betrachtung des Aufwands in Relation zum Nutzen nötig.


\subsection{Future Work}
Um die Vielfältigkeit dieses Projektes aufzuzeigen und für weitere Arbeiten zu inspirieren, ist es sinnvoll, noch offene Themen und Erweiterungen zu beleuchten. Der erste nennenswerte Punkte ist ein gezieltes Erhöhen der Robustheit des Programms. Der für uns optimale Schwellwert zur Blinzeldetektion wurde empirisch ermittelt und war für unser Anwendungsgebiet passend. Sollte man diese Voreinstellungen auf andere Ethnien testen, könnte es jedoch gegebenenfalls zu Komplikationen kommen. Um dem entgegenzuwirken, sollte der Schwellwert variabel gesetzt werden, indem ein optimaler relativer Threshold zum durchschnittlichen EAR-Wert der Person ermittelt wird. Um diesen zu ermitteln sind jedoch weitere Test und Forschung notwendig, jedoch ist dies eine Verbesserung der Anwendbarkeit des Programms.

Weiterhin kann die Wahl anderer Klassifikationsverfahren eine zusätzliche Genauigkeitsverbesserung mit sich bringen. Hierfür sind verschiedenste Versuche notwendig, um aus vielen Klassifikatoren den bestgeeigneten ermitteln zu können, wir haben uns dabei aus aufwandsmindernden Gründen auf drei Möglichkeiten beschränkt. Die Wahl einer anderen Methode könnte weiterhin den Vorteil mit sich bringen, dass man unter zusätzlicher Vergrößerung des Datensatzes einen Score zur Müdigkeit entwickeln kann. Somit verfällt eine strikte Einteilung in vordefinierte Klassen und Veränderungen der Werte über bestimmte Zeitreihen sind besser nachzuvollziehen und zu analysieren.

Um unser Anwendungsgebiet des Autofahrens weiter zu vertiefen und die Sicherheit zusätzlich zu erhöhen, ist es möglich, nicht nur die Müdigkeit zu erkennen, sondern auch Unaufmerksamkeiten zu detektieren. Von hoher Relevanz hierbei kann ein Erkennen der Zeit sein, in welcher der Fokus nicht auf den Verkehr gelegt wird. Somit erscheint eine Warnung, wenn sich der Blick zu lange dem Handy, Infotainmentsystem des Fahrzeugs oder in der weniger relevanten Fahrumgebung richtet.

Sollte eine solche Software professionelle Anwendung in einem Auto finden, so ist ein Einsatz auf sich im Auto befindender Hardware unausweichlich. Somit kann beispielsweise im Armaturenbrett eine kleine Webcam installiert sein, welche über das Boardsystem angeschlossen ist und die Warnungen können direkt über ein Cockpit, Head-Up Display oder die interne Soundanlage wiedergegeben werden. Sollten diese Schritte zukünftig genauer bedacht werden, so würde dem Weg dieser Technologie in Serienfahrzeuge nichts entgegenstehen und die Sicherheit aller Verkehrsteilnehmenden kann somit maßgeblich verbessert werden.


\subsection{Zusammenfassung}
Zusammenfassend lässt sich für die Projektarbeit sagen, dass im Rahmen des gesamten Erarbeitungsprozesses viel gelernt wurde. Einerseits konnte theoretisches Fachwissen in die Praxis umgesetzt werden, andererseits konnte man erste Erfahrungen mit der Planung und Organisation einer solchen Entwicklung sammeln. Unsere Anfangsvorstellung einer Anwendung zur Müdigkeitsdetektion über eine interne Kamera konnten wir in die Realität umsetzen und, weiter verbessert und optimiert, kann sie einen maßgeblichen Einfluss auf die Verkehrssicherheit nehmen. Wir betiteln sowohl dieses Modul, als auch unsere Gruppenarbeit als erfolgreich und sind mit unserem Ergebnis zufrieden, unter der Prämisse, dass dies ein nie endendes Thema mit stetiger Weiterentwicklung ist. Mit unserer Arbeit ist der Grundstein für mögliche weitere Arbeiten gegeben.