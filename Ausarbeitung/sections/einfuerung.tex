Die Anwendung von digitaler Bildverarbeitung und maschinellem Lernen hat in den letzten Jahren eine transformative Bedeutung erlangt und spielt eine zunehmend entscheidende Rolle in einer breiten Palette von Anwendungsgebieten. Diese Technologien ermöglichen es, visuelle Informationen präzise zu analysieren, Muster zu erkennen und darauf basierend intelligente Entscheidungen zu treffen. Die Relevanz dieser Technologien erstreckt sich über eine Vielzahl von Anwendungsgebieten, von der Medizin über die Robotik bis hin zur Automobilindustrie und zur Sicherheitstechnik.

\subsection{Aufgabenstellung}
Im Rahmen der Bachelorstudiengänge Informatik und Angewandte Informatik an der Friedrich-Schiller-Universität Jena wird die Möglichkeit geboten, das Modul Projekt Intelligente Systeme zu belegen. Da in dem Bereich Rechnersehen und Mustererkennung bereits erste Kenntnisse vermittelt wurden und sich ein Interesse ausbilden konnte, ist es naheliegend, die erlangten theoretischen Fähigkeiten mit einem Projekt in die Praxis umsetzen. Infolgedessen entschied man sich für dieses Modul, um in Gruppenarbeit aktuelle Techniken und Methodiken der digitalen Bildverarbeitung und des maschinellen Lernens weiter zu erproben. Dabei sollten bestehende Verfahren nach Stand der Kunst genutzt und sinnvoll eingesetzt werden, um ein umfassenderes Bildverarbeitungsprogramm zu entwickeln. Der Umfang dieses Projektes reicht von der Projektplanung und Konzeptionierung über die Dokumentation bis zur Vorstellung der fertigen Anwendung und der Anfertigung eines Abschlussberichts. Dabei soll man sich zu Beginn mit der Formulierung der Projektzielen nach der SMART-Methode auseinandersetzen. Anschließend sind Arbeitspakete und erste Zuständigkeiten auszuformulieren, welche in einem Gantt-Diagramm zur Planung wiederzufinden sind. Das Projekt wird in Gruppenarbeit angefertigt und bis zur Vorstellung dem Arbeitsplan gemäß entwickelt. Welche Thematik die Anwendung gezielt anspricht, ist den Gruppenmitgliedern freigestellt und unterliegt deren Interesse und Motivation.

\subsection{Motivation}
Durch die thematische Freistellung musste zu Beginn festgelegt werden, welcher Anwendungsbereich von Interesse ist. Schnell herrschte Einigkeit darüber, dass man etwas entwickeln möchte, was eine große Menge potenzieller Anwender anspricht. Das Projekt sollte demzufolge ein Ergebnis liefern, welches im Alltag anwendbar ist und bestenfalls zusätzlich einen gesellschaftlichen Nutzen aufweist, in dem für bestimmte Vorgehen die Sicherheit erhöht wird. Eine Erhöhung der Sicherheit ist im stressgeprägten Alltag, wobei Vieles schnellstmöglich vonstattengehen soll, von großer Relevanz. Vor allem im Straßenverkehr herrscht durch die hohe Grundgeschwindigkeit ein erhöhtes Risiko, weshalb kurze Unaufmerksamkeiten, durch beispielsweise Sekundenschlaf, bereits schwerwiegende Folgen mit sich ziehen können. In diesem Bereich kann die Kontrolle des Aktivitätsstatus von großer Bedeutung sein, da hiermit die Sicherheit aller Verkehrsteilnehmer positiv beeinflusst werden kann. Eine weitere Anwendung eines solchen Müdigkeitsdetektors kann in der aktiven Kontrolle ausgewählter Mitarbeiter mit sicherheitsrelevanten Jobs sein, oder auch in Online-Klassen zur Aktivitätsüberwachung der Teilnehmer. All diese Anwendungsgebiete, welche einen Fußabdruck in der Gesellschaft hinterlassen können, motivieren uns dazu, einen Müdigkeitsdetektor im Rahmen dieses Projekts zu entwickeln. Das Programm soll über eine Webcam laufen, die Person wird detektiert und verschiedene signifikante Merkmale für Müdigkeitsanzeichen werden stetig überprüft. Weiterhin sollen potenzielle Ausfälle frühzeitig erkannt und durch eine Warnung verhindert werden, somit kann ein wesentlicher Teil zur Sicherheit beigetragen werden.

\subsection{Ziele, Arbeitspakete und GANTT-Diagramm}
Optionaler Teil, eventuell auslassen, in Einführung hinten dran ballern
Aus der gewählten Thematik leiten sich somit die Ziele ab, welche mit diesem Projekt verfolgt werden sollen. Diese gemeinsamen auszuarbeiten und zu besprechen war der Start des Moduls. Hierbei ERGEBNISSE PRÄSENTIEREN
